\documentclass[11pt]{article}

    
\usepackage[breakable]{tcolorbox}
    \usepackage{parskip} % Stop auto-indenting (to mimic markdown behaviour)
    
    \usepackage{iftex}
    \ifPDFTeX
    	\usepackage[T1]{fontenc}
    	\usepackage{mathpazo}
    \else
    	\usepackage{fontspec}
    \fi

    % Basic figure setup, for now with no caption control since it's done
    % automatically by Pandoc (which extracts ![](path) syntax from Markdown).
    \usepackage{graphicx}
    % Maintain compatibility with old templates. Remove in nbconvert 6.0
    \let\Oldincludegraphics\includegraphics
    % Ensure that by default, figures have no caption (until we provide a
    % proper Figure object with a Caption API and a way to capture that
    % in the conversion process - todo).
    \usepackage{caption}
    \DeclareCaptionFormat{nocaption}{}
    \captionsetup{format=nocaption,aboveskip=0pt,belowskip=0pt}

    \usepackage[Export]{adjustbox} % Used to constrain images to a maximum size
    \adjustboxset{max size={0.9\linewidth}{0.9\paperheight}}
    \usepackage{float}
    \floatplacement{figure}{H} % forces figures to be placed at the correct location
    \usepackage{xcolor} % Allow colors to be defined
    \usepackage{enumerate} % Needed for markdown enumerations to work
    \usepackage{geometry} % Used to adjust the document margins
    \usepackage{amsmath} % Equations
    \usepackage{amssymb} % Equations
    \usepackage{textcomp} % defines textquotesingle
    % Hack from http://tex.stackexchange.com/a/47451/13684:
    \AtBeginDocument{%
        \def\PYZsq{\textquotesingle}% Upright quotes in Pygmentized code
    }
    \usepackage{upquote} % Upright quotes for verbatim code
    \usepackage{eurosym} % defines \euro
    \usepackage[mathletters]{ucs} % Extended unicode (utf-8) support
    \usepackage{fancyvrb} % verbatim replacement that allows latex
    \usepackage{grffile} % extends the file name processing of package graphics 
                         % to support a larger range
    \makeatletter % fix for grffile with XeLaTeX
    \def\Gread@@xetex#1{%
      \IfFileExists{"\Gin@base".bb}%
      {\Gread@eps{\Gin@base.bb}}%
      {\Gread@@xetex@aux#1}%
    }
    \makeatother

    % The hyperref package gives us a pdf with properly built
    % internal navigation ('pdf bookmarks' for the table of contents,
    % internal cross-reference links, web links for URLs, etc.)
    \usepackage{hyperref}
    % The default LaTeX title has an obnoxious amount of whitespace. By default,
    % titling removes some of it. It also provides customization options.
    \usepackage{titling}
    \usepackage{longtable} % longtable support required by pandoc >1.10
    \usepackage{booktabs}  % table support for pandoc > 1.12.2
    \usepackage[inline]{enumitem} % IRkernel/repr support (it uses the enumerate* environment)
    \usepackage[normalem]{ulem} % ulem is needed to support strikethroughs (\sout)
                                % normalem makes italics be italics, not underlines
    \usepackage{mathrsfs}
    

\usepackage{nopageno}



    
    % Colors for the hyperref package
    \definecolor{urlcolor}{rgb}{0,.145,.698}
    \definecolor{linkcolor}{rgb}{.71,0.21,0.01}
    \definecolor{citecolor}{rgb}{.12,.54,.11}

    % ANSI colors
    \definecolor{ansi-black}{HTML}{3E424D}
    \definecolor{ansi-black-intense}{HTML}{282C36}
    \definecolor{ansi-red}{HTML}{E75C58}
    \definecolor{ansi-red-intense}{HTML}{B22B31}
    \definecolor{ansi-green}{HTML}{00A250}
    \definecolor{ansi-green-intense}{HTML}{007427}
    \definecolor{ansi-yellow}{HTML}{DDB62B}
    \definecolor{ansi-yellow-intense}{HTML}{B27D12}
    \definecolor{ansi-blue}{HTML}{208FFB}
    \definecolor{ansi-blue-intense}{HTML}{0065CA}
    \definecolor{ansi-magenta}{HTML}{D160C4}
    \definecolor{ansi-magenta-intense}{HTML}{A03196}
    \definecolor{ansi-cyan}{HTML}{60C6C8}
    \definecolor{ansi-cyan-intense}{HTML}{258F8F}
    \definecolor{ansi-white}{HTML}{C5C1B4}
    \definecolor{ansi-white-intense}{HTML}{A1A6B2}
    \definecolor{ansi-default-inverse-fg}{HTML}{FFFFFF}
    \definecolor{ansi-default-inverse-bg}{HTML}{000000}

    % commands and environments needed by pandoc snippets
    % extracted from the output of `pandoc -s`
    \providecommand{\tightlist}{%
      \setlength{\itemsep}{0pt}\setlength{\parskip}{0pt}}
    \DefineVerbatimEnvironment{Highlighting}{Verbatim}{commandchars=\\\{\}}
    % Add ',fontsize=\small' for more characters per line
    \newenvironment{Shaded}{}{}
    \newcommand{\KeywordTok}[1]{\textcolor[rgb]{0.00,0.44,0.13}{\textbf{{#1}}}}
    \newcommand{\DataTypeTok}[1]{\textcolor[rgb]{0.56,0.13,0.00}{{#1}}}
    \newcommand{\DecValTok}[1]{\textcolor[rgb]{0.25,0.63,0.44}{{#1}}}
    \newcommand{\BaseNTok}[1]{\textcolor[rgb]{0.25,0.63,0.44}{{#1}}}
    \newcommand{\FloatTok}[1]{\textcolor[rgb]{0.25,0.63,0.44}{{#1}}}
    \newcommand{\CharTok}[1]{\textcolor[rgb]{0.25,0.44,0.63}{{#1}}}
    \newcommand{\StringTok}[1]{\textcolor[rgb]{0.25,0.44,0.63}{{#1}}}
    \newcommand{\CommentTok}[1]{\textcolor[rgb]{0.38,0.63,0.69}{\textit{{#1}}}}
    \newcommand{\OtherTok}[1]{\textcolor[rgb]{0.00,0.44,0.13}{{#1}}}
    \newcommand{\AlertTok}[1]{\textcolor[rgb]{1.00,0.00,0.00}{\textbf{{#1}}}}
    \newcommand{\FunctionTok}[1]{\textcolor[rgb]{0.02,0.16,0.49}{{#1}}}
    \newcommand{\RegionMarkerTok}[1]{{#1}}
    \newcommand{\ErrorTok}[1]{\textcolor[rgb]{1.00,0.00,0.00}{\textbf{{#1}}}}
    \newcommand{\NormalTok}[1]{{#1}}
    
    % Additional commands for more recent versions of Pandoc
    \newcommand{\ConstantTok}[1]{\textcolor[rgb]{0.53,0.00,0.00}{{#1}}}
    \newcommand{\SpecialCharTok}[1]{\textcolor[rgb]{0.25,0.44,0.63}{{#1}}}
    \newcommand{\VerbatimStringTok}[1]{\textcolor[rgb]{0.25,0.44,0.63}{{#1}}}
    \newcommand{\SpecialStringTok}[1]{\textcolor[rgb]{0.73,0.40,0.53}{{#1}}}
    \newcommand{\ImportTok}[1]{{#1}}
    \newcommand{\DocumentationTok}[1]{\textcolor[rgb]{0.73,0.13,0.13}{\textit{{#1}}}}
    \newcommand{\AnnotationTok}[1]{\textcolor[rgb]{0.38,0.63,0.69}{\textbf{\textit{{#1}}}}}
    \newcommand{\CommentVarTok}[1]{\textcolor[rgb]{0.38,0.63,0.69}{\textbf{\textit{{#1}}}}}
    \newcommand{\VariableTok}[1]{\textcolor[rgb]{0.10,0.09,0.49}{{#1}}}
    \newcommand{\ControlFlowTok}[1]{\textcolor[rgb]{0.00,0.44,0.13}{\textbf{{#1}}}}
    \newcommand{\OperatorTok}[1]{\textcolor[rgb]{0.40,0.40,0.40}{{#1}}}
    \newcommand{\BuiltInTok}[1]{{#1}}
    \newcommand{\ExtensionTok}[1]{{#1}}
    \newcommand{\PreprocessorTok}[1]{\textcolor[rgb]{0.74,0.48,0.00}{{#1}}}
    \newcommand{\AttributeTok}[1]{\textcolor[rgb]{0.49,0.56,0.16}{{#1}}}
    \newcommand{\InformationTok}[1]{\textcolor[rgb]{0.38,0.63,0.69}{\textbf{\textit{{#1}}}}}
    \newcommand{\WarningTok}[1]{\textcolor[rgb]{0.38,0.63,0.69}{\textbf{\textit{{#1}}}}}
    
    
    % Define a nice break command that doesn't care if a line doesn't already
    % exist.
    \def\br{\hspace*{\fill} \\* }
    % Math Jax compatibility definitions
    \def\gt{>}
    \def\lt{<}
    \let\Oldtex\TeX
    \let\Oldlatex\LaTeX
    \renewcommand{\TeX}{\textrm{\Oldtex}}
    \renewcommand{\LaTeX}{\textrm{\Oldlatex}}
    % Document parameters
    % Document title
    \title{Runge-Kutta Methods}
    
\date{}

    
    
    
    
% Pygments definitions
\makeatletter
\def\PY@reset{\let\PY@it=\relax \let\PY@bf=\relax%
    \let\PY@ul=\relax \let\PY@tc=\relax%
    \let\PY@bc=\relax \let\PY@ff=\relax}
\def\PY@tok#1{\csname PY@tok@#1\endcsname}
\def\PY@toks#1+{\ifx\relax#1\empty\else%
    \PY@tok{#1}\expandafter\PY@toks\fi}
\def\PY@do#1{\PY@bc{\PY@tc{\PY@ul{%
    \PY@it{\PY@bf{\PY@ff{#1}}}}}}}
\def\PY#1#2{\PY@reset\PY@toks#1+\relax+\PY@do{#2}}

\expandafter\def\csname PY@tok@w\endcsname{\def\PY@tc##1{\textcolor[rgb]{0.73,0.73,0.73}{##1}}}
\expandafter\def\csname PY@tok@c\endcsname{\let\PY@it=\textit\def\PY@tc##1{\textcolor[rgb]{0.25,0.50,0.50}{##1}}}
\expandafter\def\csname PY@tok@cp\endcsname{\def\PY@tc##1{\textcolor[rgb]{0.74,0.48,0.00}{##1}}}
\expandafter\def\csname PY@tok@k\endcsname{\let\PY@bf=\textbf\def\PY@tc##1{\textcolor[rgb]{0.00,0.50,0.00}{##1}}}
\expandafter\def\csname PY@tok@kp\endcsname{\def\PY@tc##1{\textcolor[rgb]{0.00,0.50,0.00}{##1}}}
\expandafter\def\csname PY@tok@kt\endcsname{\def\PY@tc##1{\textcolor[rgb]{0.69,0.00,0.25}{##1}}}
\expandafter\def\csname PY@tok@o\endcsname{\def\PY@tc##1{\textcolor[rgb]{0.40,0.40,0.40}{##1}}}
\expandafter\def\csname PY@tok@ow\endcsname{\let\PY@bf=\textbf\def\PY@tc##1{\textcolor[rgb]{0.67,0.13,1.00}{##1}}}
\expandafter\def\csname PY@tok@nb\endcsname{\def\PY@tc##1{\textcolor[rgb]{0.00,0.50,0.00}{##1}}}
\expandafter\def\csname PY@tok@nf\endcsname{\def\PY@tc##1{\textcolor[rgb]{0.00,0.00,1.00}{##1}}}
\expandafter\def\csname PY@tok@nc\endcsname{\let\PY@bf=\textbf\def\PY@tc##1{\textcolor[rgb]{0.00,0.00,1.00}{##1}}}
\expandafter\def\csname PY@tok@nn\endcsname{\let\PY@bf=\textbf\def\PY@tc##1{\textcolor[rgb]{0.00,0.00,1.00}{##1}}}
\expandafter\def\csname PY@tok@ne\endcsname{\let\PY@bf=\textbf\def\PY@tc##1{\textcolor[rgb]{0.82,0.25,0.23}{##1}}}
\expandafter\def\csname PY@tok@nv\endcsname{\def\PY@tc##1{\textcolor[rgb]{0.10,0.09,0.49}{##1}}}
\expandafter\def\csname PY@tok@no\endcsname{\def\PY@tc##1{\textcolor[rgb]{0.53,0.00,0.00}{##1}}}
\expandafter\def\csname PY@tok@nl\endcsname{\def\PY@tc##1{\textcolor[rgb]{0.63,0.63,0.00}{##1}}}
\expandafter\def\csname PY@tok@ni\endcsname{\let\PY@bf=\textbf\def\PY@tc##1{\textcolor[rgb]{0.60,0.60,0.60}{##1}}}
\expandafter\def\csname PY@tok@na\endcsname{\def\PY@tc##1{\textcolor[rgb]{0.49,0.56,0.16}{##1}}}
\expandafter\def\csname PY@tok@nt\endcsname{\let\PY@bf=\textbf\def\PY@tc##1{\textcolor[rgb]{0.00,0.50,0.00}{##1}}}
\expandafter\def\csname PY@tok@nd\endcsname{\def\PY@tc##1{\textcolor[rgb]{0.67,0.13,1.00}{##1}}}
\expandafter\def\csname PY@tok@s\endcsname{\def\PY@tc##1{\textcolor[rgb]{0.73,0.13,0.13}{##1}}}
\expandafter\def\csname PY@tok@sd\endcsname{\let\PY@it=\textit\def\PY@tc##1{\textcolor[rgb]{0.73,0.13,0.13}{##1}}}
\expandafter\def\csname PY@tok@si\endcsname{\let\PY@bf=\textbf\def\PY@tc##1{\textcolor[rgb]{0.73,0.40,0.53}{##1}}}
\expandafter\def\csname PY@tok@se\endcsname{\let\PY@bf=\textbf\def\PY@tc##1{\textcolor[rgb]{0.73,0.40,0.13}{##1}}}
\expandafter\def\csname PY@tok@sr\endcsname{\def\PY@tc##1{\textcolor[rgb]{0.73,0.40,0.53}{##1}}}
\expandafter\def\csname PY@tok@ss\endcsname{\def\PY@tc##1{\textcolor[rgb]{0.10,0.09,0.49}{##1}}}
\expandafter\def\csname PY@tok@sx\endcsname{\def\PY@tc##1{\textcolor[rgb]{0.00,0.50,0.00}{##1}}}
\expandafter\def\csname PY@tok@m\endcsname{\def\PY@tc##1{\textcolor[rgb]{0.40,0.40,0.40}{##1}}}
\expandafter\def\csname PY@tok@gh\endcsname{\let\PY@bf=\textbf\def\PY@tc##1{\textcolor[rgb]{0.00,0.00,0.50}{##1}}}
\expandafter\def\csname PY@tok@gu\endcsname{\let\PY@bf=\textbf\def\PY@tc##1{\textcolor[rgb]{0.50,0.00,0.50}{##1}}}
\expandafter\def\csname PY@tok@gd\endcsname{\def\PY@tc##1{\textcolor[rgb]{0.63,0.00,0.00}{##1}}}
\expandafter\def\csname PY@tok@gi\endcsname{\def\PY@tc##1{\textcolor[rgb]{0.00,0.63,0.00}{##1}}}
\expandafter\def\csname PY@tok@gr\endcsname{\def\PY@tc##1{\textcolor[rgb]{1.00,0.00,0.00}{##1}}}
\expandafter\def\csname PY@tok@ge\endcsname{\let\PY@it=\textit}
\expandafter\def\csname PY@tok@gs\endcsname{\let\PY@bf=\textbf}
\expandafter\def\csname PY@tok@gp\endcsname{\let\PY@bf=\textbf\def\PY@tc##1{\textcolor[rgb]{0.00,0.00,0.50}{##1}}}
\expandafter\def\csname PY@tok@go\endcsname{\def\PY@tc##1{\textcolor[rgb]{0.53,0.53,0.53}{##1}}}
\expandafter\def\csname PY@tok@gt\endcsname{\def\PY@tc##1{\textcolor[rgb]{0.00,0.27,0.87}{##1}}}
\expandafter\def\csname PY@tok@err\endcsname{\def\PY@bc##1{\setlength{\fboxsep}{0pt}\fcolorbox[rgb]{1.00,0.00,0.00}{1,1,1}{\strut ##1}}}
\expandafter\def\csname PY@tok@kc\endcsname{\let\PY@bf=\textbf\def\PY@tc##1{\textcolor[rgb]{0.00,0.50,0.00}{##1}}}
\expandafter\def\csname PY@tok@kd\endcsname{\let\PY@bf=\textbf\def\PY@tc##1{\textcolor[rgb]{0.00,0.50,0.00}{##1}}}
\expandafter\def\csname PY@tok@kn\endcsname{\let\PY@bf=\textbf\def\PY@tc##1{\textcolor[rgb]{0.00,0.50,0.00}{##1}}}
\expandafter\def\csname PY@tok@kr\endcsname{\let\PY@bf=\textbf\def\PY@tc##1{\textcolor[rgb]{0.00,0.50,0.00}{##1}}}
\expandafter\def\csname PY@tok@bp\endcsname{\def\PY@tc##1{\textcolor[rgb]{0.00,0.50,0.00}{##1}}}
\expandafter\def\csname PY@tok@fm\endcsname{\def\PY@tc##1{\textcolor[rgb]{0.00,0.00,1.00}{##1}}}
\expandafter\def\csname PY@tok@vc\endcsname{\def\PY@tc##1{\textcolor[rgb]{0.10,0.09,0.49}{##1}}}
\expandafter\def\csname PY@tok@vg\endcsname{\def\PY@tc##1{\textcolor[rgb]{0.10,0.09,0.49}{##1}}}
\expandafter\def\csname PY@tok@vi\endcsname{\def\PY@tc##1{\textcolor[rgb]{0.10,0.09,0.49}{##1}}}
\expandafter\def\csname PY@tok@vm\endcsname{\def\PY@tc##1{\textcolor[rgb]{0.10,0.09,0.49}{##1}}}
\expandafter\def\csname PY@tok@sa\endcsname{\def\PY@tc##1{\textcolor[rgb]{0.73,0.13,0.13}{##1}}}
\expandafter\def\csname PY@tok@sb\endcsname{\def\PY@tc##1{\textcolor[rgb]{0.73,0.13,0.13}{##1}}}
\expandafter\def\csname PY@tok@sc\endcsname{\def\PY@tc##1{\textcolor[rgb]{0.73,0.13,0.13}{##1}}}
\expandafter\def\csname PY@tok@dl\endcsname{\def\PY@tc##1{\textcolor[rgb]{0.73,0.13,0.13}{##1}}}
\expandafter\def\csname PY@tok@s2\endcsname{\def\PY@tc##1{\textcolor[rgb]{0.73,0.13,0.13}{##1}}}
\expandafter\def\csname PY@tok@sh\endcsname{\def\PY@tc##1{\textcolor[rgb]{0.73,0.13,0.13}{##1}}}
\expandafter\def\csname PY@tok@s1\endcsname{\def\PY@tc##1{\textcolor[rgb]{0.73,0.13,0.13}{##1}}}
\expandafter\def\csname PY@tok@mb\endcsname{\def\PY@tc##1{\textcolor[rgb]{0.40,0.40,0.40}{##1}}}
\expandafter\def\csname PY@tok@mf\endcsname{\def\PY@tc##1{\textcolor[rgb]{0.40,0.40,0.40}{##1}}}
\expandafter\def\csname PY@tok@mh\endcsname{\def\PY@tc##1{\textcolor[rgb]{0.40,0.40,0.40}{##1}}}
\expandafter\def\csname PY@tok@mi\endcsname{\def\PY@tc##1{\textcolor[rgb]{0.40,0.40,0.40}{##1}}}
\expandafter\def\csname PY@tok@il\endcsname{\def\PY@tc##1{\textcolor[rgb]{0.40,0.40,0.40}{##1}}}
\expandafter\def\csname PY@tok@mo\endcsname{\def\PY@tc##1{\textcolor[rgb]{0.40,0.40,0.40}{##1}}}
\expandafter\def\csname PY@tok@ch\endcsname{\let\PY@it=\textit\def\PY@tc##1{\textcolor[rgb]{0.25,0.50,0.50}{##1}}}
\expandafter\def\csname PY@tok@cm\endcsname{\let\PY@it=\textit\def\PY@tc##1{\textcolor[rgb]{0.25,0.50,0.50}{##1}}}
\expandafter\def\csname PY@tok@cpf\endcsname{\let\PY@it=\textit\def\PY@tc##1{\textcolor[rgb]{0.25,0.50,0.50}{##1}}}
\expandafter\def\csname PY@tok@c1\endcsname{\let\PY@it=\textit\def\PY@tc##1{\textcolor[rgb]{0.25,0.50,0.50}{##1}}}
\expandafter\def\csname PY@tok@cs\endcsname{\let\PY@it=\textit\def\PY@tc##1{\textcolor[rgb]{0.25,0.50,0.50}{##1}}}

\def\PYZbs{\char`\\}
\def\PYZus{\char`\_}
\def\PYZob{\char`\{}
\def\PYZcb{\char`\}}
\def\PYZca{\char`\^}
\def\PYZam{\char`\&}
\def\PYZlt{\char`\<}
\def\PYZgt{\char`\>}
\def\PYZsh{\char`\#}
\def\PYZpc{\char`\%}
\def\PYZdl{\char`\$}
\def\PYZhy{\char`\-}
\def\PYZsq{\char`\'}
\def\PYZdq{\char`\"}
\def\PYZti{\char`\~}
% for compatibility with earlier versions
\def\PYZat{@}
\def\PYZlb{[}
\def\PYZrb{]}
\makeatother


    % For linebreaks inside Verbatim environment from package fancyvrb. 
    \makeatletter
        \newbox\Wrappedcontinuationbox 
        \newbox\Wrappedvisiblespacebox 
        \newcommand*\Wrappedvisiblespace {\textcolor{red}{\textvisiblespace}} 
        \newcommand*\Wrappedcontinuationsymbol {\textcolor{red}{\llap{\tiny$\m@th\hookrightarrow$}}} 
        \newcommand*\Wrappedcontinuationindent {3ex } 
        \newcommand*\Wrappedafterbreak {\kern\Wrappedcontinuationindent\copy\Wrappedcontinuationbox} 
        % Take advantage of the already applied Pygments mark-up to insert 
        % potential linebreaks for TeX processing. 
        %        {, <, #, %, $, ' and ": go to next line. 
        %        _, }, ^, &, >, - and ~: stay at end of broken line. 
        % Use of \textquotesingle for straight quote. 
        \newcommand*\Wrappedbreaksatspecials {% 
            \def\PYGZus{\discretionary{\char`\_}{\Wrappedafterbreak}{\char`\_}}% 
            \def\PYGZob{\discretionary{}{\Wrappedafterbreak\char`\{}{\char`\{}}% 
            \def\PYGZcb{\discretionary{\char`\}}{\Wrappedafterbreak}{\char`\}}}% 
            \def\PYGZca{\discretionary{\char`\^}{\Wrappedafterbreak}{\char`\^}}% 
            \def\PYGZam{\discretionary{\char`\&}{\Wrappedafterbreak}{\char`\&}}% 
            \def\PYGZlt{\discretionary{}{\Wrappedafterbreak\char`\<}{\char`\<}}% 
            \def\PYGZgt{\discretionary{\char`\>}{\Wrappedafterbreak}{\char`\>}}% 
            \def\PYGZsh{\discretionary{}{\Wrappedafterbreak\char`\#}{\char`\#}}% 
            \def\PYGZpc{\discretionary{}{\Wrappedafterbreak\char`\%}{\char`\%}}% 
            \def\PYGZdl{\discretionary{}{\Wrappedafterbreak\char`\$}{\char`\$}}% 
            \def\PYGZhy{\discretionary{\char`\-}{\Wrappedafterbreak}{\char`\-}}% 
            \def\PYGZsq{\discretionary{}{\Wrappedafterbreak\textquotesingle}{\textquotesingle}}% 
            \def\PYGZdq{\discretionary{}{\Wrappedafterbreak\char`\"}{\char`\"}}% 
            \def\PYGZti{\discretionary{\char`\~}{\Wrappedafterbreak}{\char`\~}}% 
        } 
        % Some characters . , ; ? ! / are not pygmentized. 
        % This macro makes them "active" and they will insert potential linebreaks 
        \newcommand*\Wrappedbreaksatpunct {% 
            \lccode`\~`\.\lowercase{\def~}{\discretionary{\hbox{\char`\.}}{\Wrappedafterbreak}{\hbox{\char`\.}}}% 
            \lccode`\~`\,\lowercase{\def~}{\discretionary{\hbox{\char`\,}}{\Wrappedafterbreak}{\hbox{\char`\,}}}% 
            \lccode`\~`\;\lowercase{\def~}{\discretionary{\hbox{\char`\;}}{\Wrappedafterbreak}{\hbox{\char`\;}}}% 
            \lccode`\~`\:\lowercase{\def~}{\discretionary{\hbox{\char`\:}}{\Wrappedafterbreak}{\hbox{\char`\:}}}% 
            \lccode`\~`\?\lowercase{\def~}{\discretionary{\hbox{\char`\?}}{\Wrappedafterbreak}{\hbox{\char`\?}}}% 
            \lccode`\~`\!\lowercase{\def~}{\discretionary{\hbox{\char`\!}}{\Wrappedafterbreak}{\hbox{\char`\!}}}% 
            \lccode`\~`\/\lowercase{\def~}{\discretionary{\hbox{\char`\/}}{\Wrappedafterbreak}{\hbox{\char`\/}}}% 
            \catcode`\.\active
            \catcode`\,\active 
            \catcode`\;\active
            \catcode`\:\active
            \catcode`\?\active
            \catcode`\!\active
            \catcode`\/\active 
            \lccode`\~`\~ 	
        }
    \makeatother

    \let\OriginalVerbatim=\Verbatim
    \makeatletter
    \renewcommand{\Verbatim}[1][1]{%
        %\parskip\z@skip
        \sbox\Wrappedcontinuationbox {\Wrappedcontinuationsymbol}%
        \sbox\Wrappedvisiblespacebox {\FV@SetupFont\Wrappedvisiblespace}%
        \def\FancyVerbFormatLine ##1{\hsize\linewidth
            \vtop{\raggedright\hyphenpenalty\z@\exhyphenpenalty\z@
                \doublehyphendemerits\z@\finalhyphendemerits\z@
                \strut ##1\strut}%
        }%
        % If the linebreak is at a space, the latter will be displayed as visible
        % space at end of first line, and a continuation symbol starts next line.
        % Stretch/shrink are however usually zero for typewriter font.
        \def\FV@Space {%
            \nobreak\hskip\z@ plus\fontdimen3\font minus\fontdimen4\font
            \discretionary{\copy\Wrappedvisiblespacebox}{\Wrappedafterbreak}
            {\kern\fontdimen2\font}%
        }%
        
        % Allow breaks at special characters using \PYG... macros.
        \Wrappedbreaksatspecials
        % Breaks at punctuation characters . , ; ? ! and / need catcode=\active 	
        \OriginalVerbatim[#1,codes*=\Wrappedbreaksatpunct]%
    }
    \makeatother

    % Exact colors from NB
    \definecolor{incolor}{HTML}{303F9F}
    \definecolor{outcolor}{HTML}{D84315}
    \definecolor{cellborder}{HTML}{CFCFCF}
    \definecolor{cellbackground}{HTML}{F7F7F7}
    
    % prompt
    \makeatletter
    \newcommand{\boxspacing}{\kern\kvtcb@left@rule\kern\kvtcb@boxsep}
    \makeatother
    \newcommand{\prompt}[4]{
        \ttfamily\llap{{\color{#2}[#3]:\hspace{3pt}#4}}\vspace{-\baselineskip}
    }
    

    
    % Prevent overflowing lines due to hard-to-break entities
    \sloppy 
    % Setup hyperref package
    \hypersetup{
      breaklinks=true,  % so long urls are correctly broken across lines
      colorlinks=true,
      urlcolor=urlcolor,
      linkcolor=linkcolor,
      citecolor=citecolor,
      }
    % Slightly bigger margins than the latex defaults
    
    \geometry{verbose,tmargin=1in,bmargin=1in,lmargin=1in,rmargin=1in}
    
    

\begin{document}
    
    \maketitle
    
    

    


    \hypertarget{runge-kutta-methods}{%
\section*{Runge-Kutta Methods}\label{runge-kutta-methods}}





    The aforementioned Euler's method is the simplest single step ODE
solving method, but has a fairly large error. The Runge-Kutta methods
are more popular due to their improved accuracy, in particular 4th and
5th order methods.





    \hypertarget{outline-of-the-derivation}{%
\subsection*{Outline of the Derivation}\label{outline-of-the-derivation}}





    The idea behind Runge-Kutta is to perform integration steps using a
weighted average of Euler-like steps. The following outline \{\% cite
efferson-numerical-methods \%\} is not a full derivation of the method,
as this requires theorems outside the scope of this course.





    \hypertarget{second-order-runge-kutta}{%
\subsubsection*{Second Order
Runge-Kutta}\label{second-order-runge-kutta}}





    We shall start by looking at second order Runge-Kutta methods. We want
to solve an ODE of the form

\[
\frac{dy}{dx} = f(x, y)
\]

on the interval \([x_i, x_{i+1}]\), where \(x_{i+1} = x_i + h\), with a
given initial condition \(y(x = x_i) = y_i\). That is we wish to
determine the value of \(y(x_{i+1}) = y_{i+1}\). We start by calculating
the gradient of \(y\) at 2 places:

\begin{itemize}
\tightlist
\item
  The start of the interval: \((x_i, y_i)\)
\item
  A point inside the interval, for which we approximate the \(y\) value
  using Euler's method:
  \((x_i + \alpha h, y_i + \alpha h f(x_i, y_i))\), for some choice of
  \(\alpha\).
\end{itemize}

We then approximate the value of \(y_{i+1}\) using Euler's method with
each of these gradients:

\begin{itemize}
\tightlist
\item
  \(y_{i+1} \approx y_i + h f(x_i, y_i)\)
\item
  \(y_{i+1} \approx y_i + h f(x_i + \alpha h, y_i + \alpha h f(x_i, y_i))\)
\end{itemize}

The final approximation of \(y_{i+1}\) is calculated by taking a
weighted average of these two approximations:

\[
y_{i+1} \approx y_i + c_1 h f(x_i, y_i) + c_2 h f(x_i + \alpha h, y_i + \alpha h f(x_i, y_i) )
\]

where \(c_1 + c_2 = 1\) is required.



    \begin{center}
    \adjustimage{max size={0.9\linewidth}{0.9\paperheight}}{output_6_0.png}
    \end{center}
    { \hspace*{\fill} \\}
    


    Now, how do we go about choosing good values for \(c_1\), \(c_2\) and
\(\alpha\)? If we Taylor expand the left-hand side of the equation
above, and the last term on the right-hand side gives us the relation:

\[
\alpha = \frac{1}{2 c_2}
\]

This still gives us a free choice of one of the parameters. Two popular
choices are:

\textbf{The trapezoid rule:} \(c1 = c2 = \tfrac{1}{2}\) and
\(\alpha = 1\), which yields:

\[
y_{i+1} = y_i + \tfrac{1}{2} h \left[ f(x_i, y_i) + f(x_i + h, y_i + h f(x_i, y_i)) \right]
\]

\textbf{The midpoint rule:} \(c1 = 0\), \(c2 = 1\) and
\(\alpha = \tfrac{1}{2}\), which yields:

\[
y_{i+1} = y_i + hf\left(x_i + \tfrac{1}{2} h, y_i + \tfrac{1}{2} h f(x_i, y_i) \right)
\]

Both of these methods have an accumulated error of \(\mathcal(h^2)\), as
opposed to Euler's method with \(\mathcal(h)\)





    \hypertarget{fourth-order-runge-kutta-rk4}{%
\subsection*{Fourth Order Runge-Kutta
(RK4)}\label{fourth-order-runge-kutta-rk4}}





    As mentioned, the more popular Runge-Kutta method is the fourth order
(for which we will not cover the derivation):

\[
y_{i+1} = y_i + \tfrac{1}{6} h~ (k_1 + 2 k_2 + 2 k_3 + k_4)
\]

where the \(k\) values are the slopes:

\[
\begin{align*}
k_1 &= f(x_i, y_i)\\
k_2 &= f\left(x_i + \tfrac{1}{2}h, y_i + \tfrac{1}{2} h k_1 \right)\\
k_3 &= f\left(x_i + \tfrac{1}{2}h, y_i + \tfrac{1}{2} h k_2\right)\\
k_4 &= f(x_i + h, y_i + k_3)
\end{align*}
\]

\(k_1\) is gradient value at the left of the interval. \(k_2\) is the
gradient at the midpoint of the interval, approximated using \(k_1\).
The \(k_3\) value is the gradient at the midpoint of the interval using
\(k_2\) to approximate it. \(k_4\) is the value of the gradient at the
right end of the interval using \(k_3\) to approximate it.

This method has an accumulated error of \(\mathcal(h^4)\)





    \hypertarget{worked-example}{%
\subsubsection*{Worked Example}\label{worked-example}}





    Consider the ordinary differential equation:

\[
\frac{d y}{dx} = \frac{1}{1 + x^2}
\]

with the initial condition \(y = 1\) at \(x = 0\).

This has the exact solution:

\[
y = 1 + \arctan(x)
\]

which we can compare are results to.



    \begin{tcolorbox}[breakable, size=fbox, boxrule=1pt, pad at break*=1mm,colback=cellbackground, colframe=cellborder]
\prompt{In}{incolor}{7}{\boxspacing}
\begin{Verbatim}[commandchars=\\\{\}]
\PY{k+kn}{import} \PY{n+nn}{numpy} \PY{k}{as} \PY{n+nn}{np}
\PY{k+kn}{import} \PY{n+nn}{matplotlib}\PY{n+nn}{.}\PY{n+nn}{pyplot} \PY{k}{as} \PY{n+nn}{plt}

\PY{n}{x0}\PY{p}{,} \PY{n}{y0} \PY{o}{=} \PY{l+m+mi}{0}\PY{p}{,} \PY{l+m+mi}{1} \PY{c+c1}{\PYZsh{}initial conditions}
\PY{n}{h} \PY{o}{=} \PY{l+m+mf}{0.05}
\PY{n}{x\PYZus{}end} \PY{o}{=} \PY{l+m+mi}{1}

\PY{c+c1}{\PYZsh{}Differential equation}
\PY{k}{def} \PY{n+nf}{f}\PY{p}{(}\PY{n}{x}\PY{p}{,} \PY{n}{y}\PY{p}{)}\PY{p}{:}
    \PY{k}{return} \PY{l+m+mi}{1}\PY{o}{/}\PY{p}{(}\PY{l+m+mi}{1} \PY{o}{+} \PY{n}{x}\PY{o}{*}\PY{n}{x}\PY{p}{)}

\PY{c+c1}{\PYZsh{}Exact solution}
\PY{k}{def} \PY{n+nf}{y\PYZus{}exact}\PY{p}{(}\PY{n}{x}\PY{p}{)}\PY{p}{:}
    \PY{k}{return} \PY{l+m+mi}{1} \PY{o}{+} \PY{n}{np}\PY{o}{.}\PY{n}{arctan}\PY{p}{(}\PY{n}{x}\PY{p}{)}

\PY{c+c1}{\PYZsh{}Constructing the arrays:}
\PY{n}{x\PYZus{}arr} \PY{o}{=} \PY{n}{np}\PY{o}{.}\PY{n}{arange}\PY{p}{(}\PY{n}{x0}\PY{p}{,} \PY{n}{x\PYZus{}end} \PY{o}{+} \PY{n}{h}\PY{p}{,} \PY{n}{h}\PY{p}{)} \PY{c+c1}{\PYZsh{}make sure it goes up to and including x\PYZus{}end}

\PY{n}{y\PYZus{}arr} \PY{o}{=} \PY{n}{np}\PY{o}{.}\PY{n}{zeros}\PY{p}{(}\PY{n}{x\PYZus{}arr}\PY{o}{.}\PY{n}{shape}\PY{p}{)}
\PY{n}{y\PYZus{}arr}\PY{p}{[}\PY{l+m+mi}{0}\PY{p}{]} \PY{o}{=} \PY{n}{y0}

\PY{c+c1}{\PYZsh{}Runge\PYZhy{}Kutta method}
\PY{k}{for} \PY{n}{i}\PY{p}{,}\PY{n}{x} \PY{o+ow}{in} \PY{n+nb}{enumerate}\PY{p}{(}\PY{n}{x\PYZus{}arr}\PY{p}{[}\PY{p}{:}\PY{o}{\PYZhy{}}\PY{l+m+mi}{1}\PY{p}{]}\PY{p}{)}\PY{p}{:}
    
    \PY{c+c1}{\PYZsh{}k values}
    \PY{n}{k1} \PY{o}{=} \PY{n}{f}\PY{p}{(}\PY{n}{x}\PY{p}{,} \PY{n}{y\PYZus{}arr}\PY{p}{[}\PY{n}{i}\PY{p}{]}\PY{p}{)}
    \PY{n}{k2} \PY{o}{=} \PY{n}{f}\PY{p}{(}\PY{n}{x} \PY{o}{+} \PY{l+m+mf}{0.5}\PY{o}{*}\PY{n}{h}\PY{p}{,} \PY{n}{y\PYZus{}arr}\PY{p}{[}\PY{n}{i}\PY{p}{]} \PY{o}{+} \PY{l+m+mf}{0.5}\PY{o}{*}\PY{n}{h}\PY{o}{*}\PY{n}{k1}\PY{p}{)}
    \PY{n}{k3} \PY{o}{=} \PY{n}{f}\PY{p}{(}\PY{n}{x} \PY{o}{+} \PY{l+m+mf}{0.5}\PY{o}{*}\PY{n}{h}\PY{p}{,} \PY{n}{y\PYZus{}arr}\PY{p}{[}\PY{n}{i}\PY{p}{]} \PY{o}{+} \PY{l+m+mf}{0.5}\PY{o}{*}\PY{n}{h}\PY{o}{*}\PY{n}{k2}\PY{p}{)}
    \PY{n}{k4} \PY{o}{=} \PY{n}{f}\PY{p}{(}\PY{n}{x} \PY{o}{+} \PY{n}{h}\PY{p}{,} \PY{n}{y\PYZus{}arr}\PY{p}{[}\PY{n}{i}\PY{p}{]} \PY{o}{+} \PY{n}{k3}\PY{p}{)}
    
    \PY{c+c1}{\PYZsh{}update}
    \PY{n}{y\PYZus{}arr}\PY{p}{[}\PY{n}{i}\PY{o}{+}\PY{l+m+mi}{1}\PY{p}{]} \PY{o}{=} \PY{n}{y\PYZus{}arr}\PY{p}{[}\PY{n}{i}\PY{p}{]} \PY{o}{+} \PY{n}{h}\PY{o}{/}\PY{l+m+mi}{6}\PY{o}{*}\PY{p}{(}\PY{n}{k1} \PY{o}{+} \PY{l+m+mi}{2}\PY{o}{*}\PY{n}{k2} \PY{o}{+} \PY{l+m+mi}{2}\PY{o}{*}\PY{n}{k3} \PY{o}{+} \PY{n}{k4}\PY{p}{)}
    

\PY{c+c1}{\PYZsh{}Plotting the solution}
\PY{n}{fig}\PY{p}{,} \PY{n}{ax} \PY{o}{=} \PY{n}{plt}\PY{o}{.}\PY{n}{subplots}\PY{p}{(}\PY{p}{)}

\PY{n}{ax}\PY{o}{.}\PY{n}{plot}\PY{p}{(}\PY{n}{x\PYZus{}arr}\PY{p}{,} \PY{n}{y\PYZus{}exact}\PY{p}{(}\PY{n}{x\PYZus{}arr}\PY{p}{)}\PY{p}{,} \PY{l+s+s1}{\PYZsq{}}\PY{l+s+s1}{\PYZhy{}r}\PY{l+s+s1}{\PYZsq{}}\PY{p}{,} \PY{n}{label} \PY{o}{=} \PY{l+s+s1}{\PYZsq{}}\PY{l+s+s1}{Exact solution}\PY{l+s+s1}{\PYZsq{}}\PY{p}{,} \PY{n}{linewidth} \PY{o}{=} \PY{l+m+mi}{2}\PY{p}{)}
\PY{n}{ax}\PY{o}{.}\PY{n}{plot}\PY{p}{(}\PY{n}{x\PYZus{}arr}\PY{p}{,} \PY{n}{y\PYZus{}arr}\PY{p}{,} \PY{l+s+s1}{\PYZsq{}}\PY{l+s+s1}{\PYZhy{}\PYZhy{}k}\PY{l+s+s1}{\PYZsq{}}\PY{p}{,} \PY{n}{label} \PY{o}{=} \PY{l+s+s1}{\PYZsq{}}\PY{l+s+s1}{RK4 solution}\PY{l+s+s1}{\PYZsq{}}\PY{p}{,} \PY{n}{linewidth} \PY{o}{=} \PY{l+m+mi}{2}\PY{p}{)}
\PY{n}{ax}\PY{o}{.}\PY{n}{set\PYZus{}xlabel}\PY{p}{(}\PY{l+s+s1}{\PYZsq{}}\PY{l+s+s1}{x}\PY{l+s+s1}{\PYZsq{}}\PY{p}{,} \PY{n}{fontsize} \PY{o}{=} \PY{l+m+mi}{14}\PY{p}{)}
\PY{n}{ax}\PY{o}{.}\PY{n}{set\PYZus{}ylabel}\PY{p}{(}\PY{l+s+s1}{\PYZsq{}}\PY{l+s+s1}{y}\PY{l+s+s1}{\PYZsq{}}\PY{p}{,} \PY{n}{fontsize} \PY{o}{=} \PY{l+m+mi}{14}\PY{p}{)}

\PY{n}{ax}\PY{o}{.}\PY{n}{legend}\PY{p}{(}\PY{n}{fontsize} \PY{o}{=} \PY{l+m+mi}{14}\PY{p}{)}

\PY{n}{plt}\PY{o}{.}\PY{n}{show}\PY{p}{(}\PY{p}{)}
\end{Verbatim}
\end{tcolorbox}

    \begin{center}
    \adjustimage{max size={0.9\linewidth}{0.9\paperheight}}{output_12_0.png}
    \end{center}
    { \hspace*{\fill} \\}
    


    \hypertarget{high-order-odes}{%
\subsubsection*{High Order ODEs}\label{high-order-odes}}





    As we have discussed in a previous page, higher order ODEs can be
reduced to a collection of coupled first order ODEs, for example:

\begin{align}
\frac{d y_0}{dx} &= f_0(x, y_0, y_1, \dots, y_{n-1})\\
\frac{d y_1}{dx} &= f_1(x, y_0, y_1, \dots, y_{n-1})\\
\frac{d y_2}{dx} &= f_2(x, y_0, y_1, \dots, y_{n-1})\\
                 &\vdots \\
\frac{d y_{n-1}}{dx} &= f_{n-1}(x, y_0, y_1, \dots, y_{n-1})
\end{align}

As we have seen, the Euler's method solution for this is fairly simple.
For the RK4 method, things are slightly more complicated. We must decide
how to calculate the \(k\) values.





    \[
y_{j, i+1} = y_{j, i} + \tfrac{h}{6} (k_{1, j} + 2 k_{2, j} + 2 k_{3, j} + k_{4, j} )
\]

Note that the \(y_j\) variables are not explicitly dependent on each
other, but on the independant variable \(x\). Thus we do not have free
choice over which \(y_j\) values to use when examining another for a
particular value of \(x\). For any change in \(x\), we expect
simultenous change in all of the \(y_j\). For this reason, when
calculating the \(k_j\) values for a particular \(y_j\), we need to
consider the changes in the other \(y_l\).

\[
\begin{array}{l l l l l l l}
k_{1, j} &= f_j (x_i, &y_{0, i}~, &\dots, &y_{j, i}~, &\dots, &~y_{n-1, i}~)\\
k_{2, j} &= f_j \big(x_i + \tfrac{1}{2}h, &y_{0, i}~ + \tfrac{1}{2} k_{1, 0}~, &\dots, &y_{j, i}~ + \tfrac{1}{2} k_{1, j}, &\dots, &y_{n-1, i}~ + \tfrac{1}{2} k_{1, n-1}~ \big)\\
k_{3, j} &= f_j \big(x_i + \tfrac{1}{2}h, &y_{0, i}~ + \tfrac{1}{2} k_{2, 0}~, &\dots, &y_{j, i}~ + \tfrac{1}{2} k_{2, j}, &\dots, &y_{n-1, i}~ + \tfrac{1}{2} k_{2, n-1}~\big)\\
k_{4, j} &= f_j ( x_i + h, &y_{0, i}~ + k_{3, 0}~, &\dots, &y_{j, i}~ + k_{3, j}, &\dots, &y_{n-1, i}~ + k_{3, n-1} )\\
\end{array}
\]

This looks more complicated then it is to apply in practice. All we need
to do is vectorize the solution, as on the previous page. We can
represent all the \(y_j\) as a vector \(\vec{y}\), i.e

\[
\vec{y} = \begin{pmatrix} y_0 \\ y_1 \\ \vdots \\ y_{n-1} \end{pmatrix}
\]

the ODE can thus be represented as:

\[
\frac{d \vec{y}}{dx} = \vec{f}(x, \vec{y}) = \begin{pmatrix} f_0 (x, \vec{y}) \\ f_1 (x, \vec{y}) \\ \vdots \\ f_{n-1} (x, \vec{y}) \end{pmatrix}
\]

and an update step as:

\[
\vec{y}_{i+1} = \vec{y}_i + \tfrac{1}{6} h (\vec{k_1} + 2 \vec{k_2} + 2 \vec{k_3} + \vec{k_4})
\]

where:

\[
\vec{k_m} = \begin{pmatrix} k_{0, m} \\ k_{1, m} \\ \vdots \\ k_{n-1, m} \end{pmatrix}
\]

Note that we can write:

\[
\begin{pmatrix}
y_{0, i}~ + \tfrac{1}{2} h k_{1, 0} \\\vdots \\y_{j, i}~ + \tfrac{1}{2} h k_{1, j} \\ \dots \\ y_{n-1, i}~ + \tfrac{1}{2} h k_{1, n-1}
\end{pmatrix}
= \vec{y}_i + \tfrac{1}{2} h \vec{k1}
\]

with this in mind, we can simply write the \(k\) values as:

\begin{align*}
\vec{k_1} &= \vec{f}(x_i, \vec{y}_i)\\
\vec{k_2} &= \vec{f}\left(x_i + \tfrac{1}{2} h, \vec{y}_i + \tfrac{1}{2} h ~ \vec{k_1} \right)\\
\vec{k_3} &= \vec{f}\left(x_i + \tfrac{1}{2} h, \vec{y}_i + \tfrac{1}{2} h ~ \vec{k_2} \right)\\
\vec{k_4} & = \vec{f}\left(x_i + h, \vec{y}_i + h ~ \vec{k_3} \right)
\end{align*}





    \hypertarget{worked-example}{%
\paragraph{Worked Example}\label{worked-example}}





    Consider the third order differential equation:

\[
\frac{d^{4}y}{dx^4} = -12 x y - 4 x^2 \frac{d y}{dx}
\]

with the initial conditions: \(y(x = 0) = 0\), \(y^\prime(0) = 0\) and
\(y^{\prime\prime}(0) = 2\).

This has an exact solution of:

\[
y(x) = e^{-x^2}
\]

which we shall use to test our numerical result.

We shall solve this up to \(x = 5\) with steps of size \(h = 0.1\).

First we reduce this to a system of first order equations by introducing
the variables \(y_0(x) = y(x)\), \(y_1(x) = y^\prime(x)\) and
\(y_2(x) = y^{\prime\prime}(x)\):

\begin{align*}
\frac{d y_0}{dx} &= y_1\\
\frac{d y_1}{dx} &= y_2\\
\frac{d y_2}{dx} &= -12 x y_0 - 4 x^2 y_1
\end{align*}



    \begin{tcolorbox}[breakable, size=fbox, boxrule=1pt, pad at break*=1mm,colback=cellbackground, colframe=cellborder]
\prompt{In}{incolor}{25}{\boxspacing}
\begin{Verbatim}[commandchars=\\\{\}]
\PY{k+kn}{import} \PY{n+nn}{numpy} \PY{k}{as} \PY{n+nn}{np}
\PY{k+kn}{import} \PY{n+nn}{matplotlib}\PY{n+nn}{.}\PY{n+nn}{pyplot} \PY{k}{as} \PY{n+nn}{plt}

\PY{n}{x0}\PY{p}{,} \PY{n}{y0} \PY{o}{=} \PY{l+m+mi}{0}\PY{p}{,} \PY{p}{[}\PY{l+m+mi}{0}\PY{p}{,} \PY{l+m+mi}{0}\PY{p}{,} \PY{l+m+mi}{2}\PY{p}{]} \PY{c+c1}{\PYZsh{}initial conditions}
\PY{n}{h} \PY{o}{=} \PY{l+m+mf}{0.1}
\PY{n}{x\PYZus{}end} \PY{o}{=} \PY{l+m+mi}{5}

\PY{k}{def} \PY{n+nf}{f}\PY{p}{(}\PY{n}{x}\PY{p}{,} \PY{n}{y}\PY{p}{)}\PY{p}{:}
    \PY{c+c1}{\PYZsh{}Important! This must return an array!}
    \PY{k}{return} \PY{n}{np}\PY{o}{.}\PY{n}{array}\PY{p}{(}\PY{p}{[}
        \PY{n}{y}\PY{p}{[}\PY{l+m+mi}{1}\PY{p}{]}\PY{p}{,}
        \PY{n}{y}\PY{p}{[}\PY{l+m+mi}{2}\PY{p}{]}\PY{p}{,}
        \PY{o}{\PYZhy{}}\PY{l+m+mi}{12}\PY{o}{*}\PY{n}{x}\PY{o}{*}\PY{n}{y}\PY{p}{[}\PY{l+m+mi}{0}\PY{p}{]} \PY{o}{\PYZhy{}} \PY{l+m+mi}{4}\PY{o}{*}\PY{n}{x}\PY{o}{*}\PY{n}{x}\PY{o}{*}\PY{n}{y}\PY{p}{[}\PY{l+m+mi}{1}\PY{p}{]}
    \PY{p}{]}\PY{p}{)}

\PY{k}{def} \PY{n+nf}{y\PYZus{}exact}\PY{p}{(}\PY{n}{x}\PY{p}{)}\PY{p}{:}
    \PY{k}{return} \PY{n}{np}\PY{o}{.}\PY{n}{sin}\PY{p}{(}\PY{n}{x}\PY{o}{*}\PY{n}{x}\PY{p}{)}

\PY{c+c1}{\PYZsh{}Constructing the arrays:}
\PY{n}{x\PYZus{}arr} \PY{o}{=} \PY{n}{np}\PY{o}{.}\PY{n}{arange}\PY{p}{(}\PY{n}{x0}\PY{p}{,} \PY{n}{x\PYZus{}end} \PY{o}{+} \PY{n}{h}\PY{p}{,} \PY{n}{h}\PY{p}{)} \PY{c+c1}{\PYZsh{}make sure it goes up to and including x\PYZus{}end}

\PY{n}{y\PYZus{}arr} \PY{o}{=} \PY{n}{np}\PY{o}{.}\PY{n}{zeros}\PY{p}{(}\PY{p}{(}\PY{n}{x\PYZus{}arr}\PY{o}{.}\PY{n}{size}\PY{p}{,} \PY{n+nb}{len}\PY{p}{(}\PY{n}{y0}\PY{p}{)}\PY{p}{)}\PY{p}{)}
\PY{n}{y\PYZus{}arr}\PY{p}{[}\PY{l+m+mi}{0}\PY{p}{,} \PY{p}{:}\PY{p}{]} \PY{o}{=} \PY{n}{y0}

\PY{c+c1}{\PYZsh{}Runge\PYZhy{}Kutta method}
\PY{k}{for} \PY{n}{i}\PY{p}{,}\PY{n}{x} \PY{o+ow}{in} \PY{n+nb}{enumerate}\PY{p}{(}\PY{n}{x\PYZus{}arr}\PY{p}{[}\PY{p}{:}\PY{o}{\PYZhy{}}\PY{l+m+mi}{1}\PY{p}{]}\PY{p}{)}\PY{p}{:}
    \PY{n}{y} \PY{o}{=} \PY{n}{y\PYZus{}arr}\PY{p}{[}\PY{n}{i}\PY{p}{,}\PY{p}{:}\PY{p}{]}
    
    \PY{c+c1}{\PYZsh{}k values}
    \PY{n}{k1} \PY{o}{=} \PY{n}{f}\PY{p}{(}\PY{n}{x}\PY{p}{,} \PY{n}{y}\PY{p}{)}
    \PY{n}{k2} \PY{o}{=} \PY{n}{f}\PY{p}{(}\PY{n}{x} \PY{o}{+} \PY{l+m+mf}{0.5}\PY{o}{*}\PY{n}{h}\PY{p}{,} \PY{n}{y} \PY{o}{+} \PY{l+m+mf}{0.5}\PY{o}{*}\PY{n}{h}\PY{o}{*}\PY{n}{k1}\PY{p}{)}
    \PY{n}{k3} \PY{o}{=} \PY{n}{f}\PY{p}{(}\PY{n}{x} \PY{o}{+} \PY{l+m+mf}{0.5}\PY{o}{*}\PY{n}{h}\PY{p}{,} \PY{n}{y} \PY{o}{+} \PY{l+m+mf}{0.5}\PY{o}{*}\PY{n}{h}\PY{o}{*}\PY{n}{k2}\PY{p}{)}
    \PY{n}{k4} \PY{o}{=} \PY{n}{f}\PY{p}{(}\PY{n}{x} \PY{o}{+} \PY{n}{h}\PY{p}{,} \PY{n}{y} \PY{o}{+} \PY{n}{h}\PY{o}{*}\PY{n}{k3}\PY{p}{)}
    
    \PY{c+c1}{\PYZsh{}update}
    \PY{n}{y\PYZus{}arr}\PY{p}{[}\PY{n}{i}\PY{o}{+}\PY{l+m+mi}{1}\PY{p}{,} \PY{p}{:}\PY{p}{]} \PY{o}{=} \PY{n}{y} \PY{o}{+} \PY{n}{h}\PY{o}{/}\PY{l+m+mi}{6}\PY{o}{*}\PY{p}{(}\PY{n}{k1} \PY{o}{+} \PY{l+m+mi}{2}\PY{o}{*}\PY{n}{k2} \PY{o}{+} \PY{l+m+mi}{2}\PY{o}{*}\PY{n}{k3} \PY{o}{+} \PY{n}{k4}\PY{p}{)}
    

\PY{c+c1}{\PYZsh{}Plotting the solution}
\PY{n}{fig}\PY{p}{,} \PY{n}{ax} \PY{o}{=} \PY{n}{plt}\PY{o}{.}\PY{n}{subplots}\PY{p}{(}\PY{p}{)}

\PY{n}{ax}\PY{o}{.}\PY{n}{plot}\PY{p}{(}\PY{n}{x\PYZus{}arr}\PY{p}{,} \PY{n}{y\PYZus{}exact}\PY{p}{(}\PY{n}{x\PYZus{}arr}\PY{p}{)}\PY{p}{,} \PY{l+s+s1}{\PYZsq{}}\PY{l+s+s1}{r\PYZhy{}}\PY{l+s+s1}{\PYZsq{}}\PY{p}{,} \PY{n}{linewidth} \PY{o}{=} \PY{l+m+mi}{2}\PY{p}{)}
\PY{n}{ax}\PY{o}{.}\PY{n}{plot}\PY{p}{(}\PY{n}{x\PYZus{}arr}\PY{p}{,} \PY{n}{y\PYZus{}arr}\PY{p}{[}\PY{p}{:}\PY{p}{,} \PY{l+m+mi}{0}\PY{p}{]}\PY{p}{,} \PY{l+s+s1}{\PYZsq{}}\PY{l+s+s1}{k\PYZhy{}\PYZhy{}}\PY{l+s+s1}{\PYZsq{}}\PY{p}{,} \PY{n}{linewidth} \PY{o}{=} \PY{l+m+mi}{2}\PY{p}{)}
\PY{n}{ax}\PY{o}{.}\PY{n}{set\PYZus{}xlabel}\PY{p}{(}\PY{l+s+s1}{\PYZsq{}}\PY{l+s+s1}{x}\PY{l+s+s1}{\PYZsq{}}\PY{p}{,} \PY{n}{fontsize} \PY{o}{=} \PY{l+m+mi}{14}\PY{p}{)}
\PY{n}{ax}\PY{o}{.}\PY{n}{set\PYZus{}ylabel}\PY{p}{(}\PY{l+s+s1}{\PYZsq{}}\PY{l+s+s1}{y}\PY{l+s+s1}{\PYZsq{}}\PY{p}{,} \PY{n}{fontsize} \PY{o}{=} \PY{l+m+mi}{14}\PY{p}{)}

\PY{n}{plt}\PY{o}{.}\PY{n}{show}\PY{p}{(}\PY{p}{)}
\end{Verbatim}
\end{tcolorbox}

    \begin{center}
    \adjustimage{max size={0.9\linewidth}{0.9\paperheight}}{output_18_0.png}
    \end{center}
    { \hspace*{\fill} \\}
    


    \hypertarget{references}{%
\subsection*{References}\label{references}}

\{\% bibliography --cited \%\}




    % Add a bibliography block to the postdoc
    
    

    
\end{document}
